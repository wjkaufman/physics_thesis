%!TEX root=thesis.tex
\chapter{Conclusion and Future Work}

Hamiltonian engineering encompasses a variety of important problems in quantum physics and quantum control. For spin systems in particular, the ability to decouple dipolar interactions is desirable for improving spectroscopy and increasing spin coherence times.
Existing approaches to Hamiltonian engineering via average Hamiltonian theory have been incredibly effective. The CORY48 pulse sequence was designed analytically to decouple all interactions to second order, and is robust to common errors in spin systems \cite{CORY1990205}. There is still room for improvement in increasing coherence times, though, and it is unclear whether analytical AHT-based methods can be taken further. The higher-order terms in the Magnus expansion contribute to the average Hamiltonian but are increasingly computationally expensive to calculate. Other approaches to Hamiltonian engineering may prove to be more promising.

Reinforcement learning (RL) algorithms have expanded dramatically over the past decade into a broad array of problems. The AlphaZero algorithm originally designed to play chess, shogi, and Go was used to develop pulse sequences for decoupling all interactions ($\overline{H} = 0$) in a spin system. The RL algorithm by itself was only effective at identifying high-fidelity pulse sequences for short sequences. By introducing additional constraints to the algorithm's tree search, high-fidelity pulse sequences were found for all sequence lengths considered. And by introducing common errors in the spin system simulations during training, the algorithm found pulse sequences that were robust to those errors. However, the RL-based pulse sequences did not outperform the CORY48 pulse sequence either in computational simulations or experiment.

% TODO talk about this somewhere?
% sparsity of rewards (only at end of episode, so doesn't get immediate feedback)
% discontinuous reward landscape, sparsity of high rewards (most pulse sequences suck, even when constrained with AHT0)

Although the current RL implementation failed to outperform existing analytical approaches, there are several factors that may be considered for improvement.
First, the hyperparameters used for the AlphaZero algorithm were largely unchanged from the original publication, with the exception of the exploration noise added during trees search. Hyperparameter tuning specifically for Hamiltonian engineering may improve the algorithm's performance and increase the fidelities of the resulting pulse sequences. Specifically, changing the exploration rate and the training rate relative to data collection from tree search may be fruitful paths to explore.

In addition to optimizing the RL algorithm, using a different set of constraints in the tree search may be beneficial. The $6\tau$ total decoupling constraint that was implemented is not followed in other pulse sequences, including CORY48, so it is possible that decoupling all interactions every $6\tau$ is too strong a constraint on pulse sequences. Including constraints to satisfy higher-order terms in the average Hamiltonian would also be beneficial.

% TODO there's still the policy function, basically unexplored
% could be interesting to see what actions it thinks are useful,
% and see if it reverse-engineers AHT...
