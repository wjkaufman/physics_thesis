%!TEX root=thesis.tex
\chapter{Background}

\section{Motivation}

Some physicists believe that quantum physics is on the cusp of a ``second quantum revolution''\cite{quantum-rev}. The ability to probe and control quantum systems at the individual level, as opposed to using emergent phenomena that arise due to quantum mechanics, has potential to radically transform many different areas, from sensing problems to quantum computing.

% TODO describe motivation, spectral lines, spin coherence

% intro to quantum ideas
% TODO delete this subsection?
\section{A Few Foundations from Quantum Mechanics}

% state of system <-> density operator
Given a quantum system, a general state can be represented by a density operator $\rho(t)$. The density operator encapsulates all known information about the quantum system, specifically the expectation values associated with well-defined observable. For an observable $A$, the expectation is given by
\[
\langle A \rangle = \Tr{\rho A}
\]

The time evolution of the density operator is given by
\begin{equation}\label{eq:density-time}
    \rho(t) = U(t) \rho(0) U(t)^\dagger
\end{equation}
where the unitary operator $U$ is the ``propagator'' defined by
\begin{equation}\label{eq:propagator-de}
    i\hbar \ddt{U(t)} = H(t) U(t), U(0) = \identity
\end{equation}
% TODO decide to include this?
The two equations above can be combined to yield the Liouville-Von Neumann equation, which explicitly shows the density operator's dependence on the system Hamiltonian.
\begin{equation}
    i\hbar \ddt{\rho(t)} = [H(t), \rho(t)]
\end{equation}

When the Hamiltonian is time-independent or if the Hamiltonian commutes with itself at different points in time (i.e. $[H(t_1), H(t_2)] = 0$), equation~\ref{eq:propagator-de} can be solved to obtain the propagator
\[
U(t) = \text{exp}\left[ {-i/\hbar \int_0^t H(t') dt'} \right]
\]
However, if the Hamiltonian is time-dependent and doesn't commute with itself at different times, then the above equation is not valid. This difficulty with time-dependent Hamiltonians and methods with dealing with them are further discussed in~\ref{sec:AHT}.

% TODO add section on toggling frames, how that works... (How does it work?)
\lipsum[1]

% quantum control and quantum sensing
Two important active fields of research are quantum control and quantum sensing. Quantum control investigates the ability to transform an initial state $\rho(0)$ to a specified final state and time $\rho(T)$, or the ability to create a specified propagator $U(T)$, using some set of controls on the system.\cite{Dong_2010} Systems can be controlled by manipulating parameters in the Hamiltonian, such as the strength or orientation of an external magnetic field.
Quantum sensing leverages quantized energy levels, coherence, or entanglement properties of systems to precisely measure physical quantities. To measure the desired quantity (such as the strength of an external magnetic field), it is sometimes necessary to minimize the other interactions present in the system. For both control and sensing problems, engineering a specific Hamiltonian is an important step.

\section{Spin-1/2 Systems and Nuclear Magnetic Resonance}

A spin-1/2 particle (such as an electron or proton) has two possible values for its intrinsic angular momentum or ``spin,'' which makes it one of the simplest quantum systems to study. The corresponding Hilbert Space for a single spin-1/2 particle has dimension two.
Nuclear magnetic resonance (NMR) is a technique % TODO continue describing what NMR is
% TODO Discuss NMR system, general overview

\begin{align}\label{eq:nmr-ham}
    H_\text{total}(t) &= H_0 + H_\text{CS} + H_D + H_\text{rf}(t) \\
    H_0 &= \sum_i \omega_i I_z^{(i)} \\
    H_\text{CS} &= ???????????????? \\
    H_D &= \sum_{i,j} d_{ij} \left( 3I_z^{(i)}I_z^{(j)} - \mathbf{I^{(i)}} \cdot \mathbf{I^{(j)}} \right)
\end{align}

\lipsum[2]

\section{Quantum Control}

% TODO q control framework, Hsys, Hcontrol

% TODO different applications (state preparation, dynamics, bath engineering)
\lipsum[1]

\subsection{Hamiltonian Engineering}

\lipsum[1-4]

% \subsection{Unitary Implementation}
% % TODO do I want to include this?
%
% \lipsum[1-3]

\section{Average Hamiltonian Theory}\label{sec:AHT}

\lipsum[1-2]

\subsection{Magnus Expansion}

\lipsum[1-2]

\subsection{Applications to Quantum Control}

\lipsum[1-5]

% \section{GRAPE}
% % TODO include?
% \lipsum[1-5]

\section{Examples}

\lipsum[1]

\subsection{WHH-4 Pulse Sequence}

\lipsum[1-3]

\subsection{CORY-48 Pulse Sequence}

\lipsum[1-3]

% TODO include figure of average correlation for FID, WHH, CORY

% \subsection{GRAPE Shaped Pulses}
% % TODO include?
% \lipsum[1-3]

\section{Reinforcement Learning}

\lipsum[1-3]

\subsection{Reinforcement Learning Taxonomy}

\lipsum[1-5]

\subsection{Algorithm Examples}

% TODO describe AlphaZero algorithm, others I tried?

\lipsum[1-5]
