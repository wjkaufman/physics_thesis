%!TEX root=thesis.tex
\chapter{Background}

% TODO delete this subsection, for outline only
\subsection{Foundations of Quantum Mechanics}


% intro to quantum ideas

Some physicists believe that quantum physics is on the cusp of ``second quantum revolution''\cite{quantum-rev}. The ability to probe and control quantum systems at the individual level, as opposed to using emergent phenomena that arise due to quantum mechanics, has potential to radically transform many different areas, from sensing problems to quantum computing.

% new section?
The fundamentals of quantum mechanics are comprised of postulates that describe how systems are represented, how measurement behaves, and how systems evolve in time.
% TODO
% state of system <-> density operator (write when I learn about)
Given a density operator $\rho$ for some quantum system, the time evolution of $\rho$ is given by the Liouville-Von Neumann equation
\begin{equation}
    i\hbar \ddt{\rho} = [\hham, \rho]
\end{equation}
Which can be solved to get the time-dependent density operator
\begin{equation}\label{eq:density-time}
    \rho(t) = U(t) \rho(t=0) U(t)^\dagger
\end{equation}
where $U$ is the ``propagator'' defined by
\begin{equation}\label{eq:propagator-de}
    i\hbar \ddt{U} = \hham U, U(t=0) = \identity
\end{equation}

When the Hamiltonian is independent of time, equation~\ref{eq:propagator-de} can be easily solved to obtain the propagator
\[
U(t) = e^{-i/\hbar \hham t}
\]
However, when the Hamiltonian depends on time, the operator will not necessarily commute with itself, so the equation
\[
U'(t) = e^{\int_0^t \hham(t') dt'}
\]
is not valid. This difficulty with time-dependent Hamiltonians is further discussed in~\ref{sec:AHT}.


% quantum control idea

\section{Average Hamiltonian Theory}\label{sec:AHT}

\subsection{Time-Dependent Hamiltonians}

\subsection{Magnus Expansion}

% asdf

\section{Existing Pulse Sequences}

\subsection{WAHUHA 4-Pulse Sequence}

\subsubsection{Derivation of Average Hamiltonian}

\subsection{MREV 8-Pulse Sequence}



% sadf

\section{Machine Learning Techniques}


\subsection{Reinforcement Learning}


\subsection{Deep Learning and Applications to RL}


% asdf
