%!TEX root=thesis.tex
\chapter{Background}

% TODO delete this subsection, for outline only
\section{Foundations of Quantum Mechanics}


% intro to quantum ideas

Some physicists believe that quantum physics is on the cusp of a ``second quantum revolution''\cite{quantum-rev}. The ability to probe and control quantum systems at the individual level, as opposed to using emergent phenomena that arise due to quantum mechanics, has potential to radically transform many different areas, from sensing problems to quantum computing.

% state of system <-> density operator
Given a quantum system, a general state can be represented by a density operator $\rho(t)$. The density operator encapsulates all known information about the quantum system, specifically the expectation values associated with well-defined observable. For an observable $A$, the expectation is given by
\[
\langle A \rangle = \Tr{\rho A}
\]

The time evolution of the density operator is given by
\begin{equation}\label{eq:density-time}
    \rho(t) = U(t) \rho(0) U(t)^\dagger
\end{equation}
where the unitary operator $U$ is the ``propagator'' defined by
\begin{equation}\label{eq:propagator-de}
    i\hbar \ddt{U(t)} = H(t) U(t), U(0) = \identity
\end{equation}
% TODO decide to include this?
The two equations above can be combined to yield the Liouville-Von Neumann equation, which explicitly shows the density operator's dependence on the system Hamiltonian.
\begin{equation}
    i\hbar \ddt{\rho(t)} = [H(t), \rho(t)]
\end{equation}

When the Hamiltonian is time-independent or if the Hamiltonian commutes with itself at different points in time (i.e. $[H(t_1), H(t_2)] = 0$), equation~\ref{eq:propagator-de} can be solved to obtain the propagator
\[
U(t) = \text{exp}\left[ {-i/\hbar \int_0^t H(t') dt'} \right]
\]
However, if the Hamiltonian is time-dependent and doesn't commute with itself at different times, then the above equation is not valid. This difficulty with time-dependent Hamiltonians and methods with dealing with them are further discussed in~\ref{sec:AHT}.

% quantum control and quantum sensing
Two important active fields of research are quantum control and quantum sensing. Quantum control involves
% TODO keep explaining this, high-level

\section{NMR Systems}

% TODO Discuss NMR system, general overview

\begin{align}\label{eq:nmr-ham}
    H_\text{total}(t) &= H_0 + H_\text{CS} + H_D + H_\text{rf}(t) \\
    H_0 &= \sum_i \omega_i I_z^{(i)} \\
    H_\text{CS} &= ???????????????? \\
    H_D &= \sum_{i,j} d_{ij} \left( 3I_z^{(i)}I_z^{(j)} - \mathbf{I^{(i)}} \cdot \mathbf{I^{(j)}} \right)
\end{align}

\section{Average Hamiltonian Theory}\label{sec:AHT}

\subsection{Time-Dependent Hamiltonians}

\subsection{Magnus Expansion}

% asdf

\section{Existing Pulse Sequences}

\subsection{WAHUHA 4-Pulse Sequence}

\subsubsection{Derivation of Average Hamiltonian}

\subsection{MREV 8-Pulse Sequence}



% sadf

\section{Machine Learning Techniques}


\subsection{Reinforcement Learning}


\subsection{Deep Learning and Applications to RL}


% asdf
