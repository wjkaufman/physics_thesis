%!TEX output_directory=output
\documentclass{report}

\usepackage{kaufman}

\title{Pulse Sequence Design using Machine Learning Techniques for Hamiltonian Engineering}
\author{Will Kaufman \\ Physics Department, Dartmouth College\thanks{Thesis Advisor Chandrasekhar Ramanathan}}

\begin{document}
\maketitle

\begin{abstract}

% high level

% machine learning for optimizing pulse sequences designed for Hamiltonian engineering
%
% studying the solution space for unitary operators, including metrics on the space and ideas of robustness
%
%
Quantum simulation permits the study of quantum systems that are difficult to observe experimentally and intractable to simulate on classical computers. To perform quantum simulations, the ``effective Hamiltonian'' must be engineered to match the desired quantum system.
% include quantum simulation as high-level motivation
% TODO fill in
In a variety of quantum systems methods have been developed for altering the effective Hamiltonian.
%For nuclear or electronic systems, % need to specify?
In some cases, as in nuclear or electronic spin systems, applying a time-dependent external magnetic field can be used to alter the Hamiltonian, allowing the system to evolve under a desired ``effective'' Hamiltonian that may eliminate the effects of certain interactions or lead to new phenomena not typically observed in nature. Average Hamiltonian Theory provides a framework in which the effective Hamiltonian can be built by a periodic sequence of magnetic field pulses. The system, observed at ``window times'' within each period appears to evolve under the average effective Hamiltonian.
% my project
In this thesis, machine learning techniques for optimizing pulse sequences will be studied. This includes exploring different metrics to quantify the fidelity of unitary operators, optimizing over the parameter space for pulse sequences to obtain the desired Hamiltonian, as well as understanding the robustness of solutions. Simulations of small-scale quantum systems will primarily be used to explore these techniques, and extensions to experimental systems with large ensembles of spins will also be considered.
% scope of project
% ???
% who cares?
This work will improve our ability to control and measure quantum systems, with applications in solid state NMR, quantum sensing, and quantum computing.

\end{abstract}

\tableofcontents

%!TEX root=thesis.tex
\chapter{Background}

\section{Motivation}

Some physicists believe that quantum physics is on the cusp of a ``second quantum revolution''\cite{quantum-rev}. The ability to probe and control quantum systems at the individual level, as opposed to using emergent phenomena that arise due to quantum mechanics, has potential to radically transform many different areas, from sensing problems to quantum computing.

% TODO describe motivation, spectral lines, spin coherence

% intro to quantum ideas
% TODO delete this subsection?
\section{A Few Foundations from Quantum Mechanics}

% state of system <-> density operator
Given a quantum system, a general state can be represented by a density operator $\rho(t)$. The density operator encapsulates all known information about the quantum system, specifically the expectation values associated with well-defined observable. For an observable $A$, the expectation is given by
\[
\langle A \rangle = \Tr{\rho A}
\]

The time evolution of the density operator is given by
\begin{equation}\label{eq:density-time}
    \rho(t) = U(t) \rho(0) U(t)^\dagger
\end{equation}
where the unitary operator $U$ is the ``propagator'' defined by
\begin{equation}\label{eq:propagator-de}
    i\hbar \ddt{U(t)} = H(t) U(t), U(0) = \identity
\end{equation}
% TODO decide to include this?
The two equations above can be combined to yield the Liouville-Von Neumann equation, which explicitly shows the density operator's dependence on the system Hamiltonian.
\begin{equation}
    i\hbar \ddt{\rho(t)} = [H(t), \rho(t)]
\end{equation}

When the Hamiltonian is time-independent or if the Hamiltonian commutes with itself at different points in time (i.e. $[H(t_1), H(t_2)] = 0$), equation~\ref{eq:propagator-de} can be solved to obtain the propagator
\[
U(t) = \text{exp}\left[ {-i/\hbar \int_0^t H(t') dt'} \right]
\]
However, if the Hamiltonian is time-dependent and doesn't commute with itself at different times, then the above equation is not valid. This difficulty with time-dependent Hamiltonians and methods with dealing with them are further discussed in~\ref{sec:AHT}.

% TODO add section on toggling frames, how that works... (How does it work?)
\lipsum[1]

% quantum control and quantum sensing
Two important active fields of research are quantum control and quantum sensing. Quantum control investigates the ability to transform an initial state $\rho(0)$ to a specified final state and time $\rho(T)$, or the ability to create a specified propagator $U(T)$, using some set of controls on the system.\cite{Dong_2010} Systems can be controlled by manipulating parameters in the Hamiltonian, such as the strength or orientation of an external magnetic field.
Quantum sensing leverages quantized energy levels, coherence, or entanglement properties of systems to precisely measure physical quantities. To measure the desired quantity (such as the strength of an external magnetic field), it is sometimes necessary to minimize the other interactions present in the system. For both control and sensing problems, engineering a specific Hamiltonian is an important step.

\section{Spin-1/2 Systems and Nuclear Magnetic Resonance}

A spin-1/2 particle (such as an electron or proton) has two possible values for its intrinsic angular momentum or ``spin,'' which makes it one of the simplest quantum systems to study. The corresponding Hilbert Space for a single spin-1/2 particle has dimension two.
Nuclear magnetic resonance (NMR) is a technique % TODO continue describing what NMR is
% TODO Discuss NMR system, general overview

\begin{align}\label{eq:nmr-ham}
    H_\text{total}(t) &= H_0 + H_\text{CS} + H_D + H_\text{rf}(t) \\
    H_0 &= \sum_i \omega_i I_z^{(i)} \\
    H_\text{CS} &= ???????????????? \\
    H_D &= \sum_{i,j} d_{ij} \left( 3I_z^{(i)}I_z^{(j)} - \mathbf{I^{(i)}} \cdot \mathbf{I^{(j)}} \right)
\end{align}

\lipsum[2]

\section{Quantum Control}

% TODO q control framework, Hsys, Hcontrol

% TODO different applications (state preparation, dynamics, bath engineering)
\lipsum[1]

\subsection{Hamiltonian Engineering}

\lipsum[1-4]

% \subsection{Unitary Implementation}
% % TODO do I want to include this?
%
% \lipsum[1-3]

\section{Average Hamiltonian Theory}\label{sec:AHT}

\lipsum[1-2]

\subsection{Magnus Expansion}

\lipsum[1-2]

\subsection{Applications to Quantum Control}

\lipsum[1-5]

% \section{GRAPE}
% % TODO include?
% \lipsum[1-5]

\section{Examples}

\lipsum[1]

\subsection{WHH-4 Pulse Sequence}

\lipsum[1-3]

\subsection{CORY-48 Pulse Sequence}

\lipsum[1-3]

% TODO include figure of average correlation for FID, WHH, CORY

% \subsection{GRAPE Shaped Pulses}
% % TODO include?
% \lipsum[1-3]

\section{Reinforcement Learning}

\lipsum[1-3]

\subsection{Reinforcement Learning Taxonomy}

\lipsum[1-5]

\subsection{Algorithm Examples}

% TODO describe AlphaZero algorithm, others I tried?

\lipsum[1-5]


\chapter{Fidelity Metrics for Unitary Operators}

% motivation for metrics on unitary operators
% diff metrics ?
% qualitative evaluation
% constraints on metrics (physical constraints on unitaries, i.e. B field limits)

\section{Operators in Quantum Mechanics}



\subsection{Relevance to Experimental Results}



\section{Propagator Fidelity Metrics}



\section{State-Dependent Fidelity Metrics}


\chapter{Optimization Problem for Pulse Sequence Design}

% physical constraints
% parameters for sequences


\chapter{Results}

% problems: refocusing, quantum sensing
% discussion of results


%!TEX root=thesis.tex
\chapter{Conclusion and Future Work}

Hamiltonian engineering encompasses a variety of important problems in quantum physics and quantum control. For spin systems in particular, the ability to decouple dipolar interactions is desirable for improving spectroscopy and increasing spin coherence times.
Existing approaches to Hamiltonian engineering via average Hamiltonian theory have been effective. The CORY48 pulse sequence was designed analytically using AHT to decouple all interactions to second order, and is robust to common errors in spin systems (\cite{CORY1990205}). However, there is still room for improvement in increasing coherence times, and it is unclear whether analytical AHT-based methods can be taken further because the higher-order terms in the Magnus expansion become increasingly computationally expensive to calculate. Other approaches to Hamiltonian engineering may prove to be more promising.

Reinforcement learning (RL) algorithms have expanded dramatically over the past decade into a broad array of problems. The AlphaZero algorithm was used to develop pulse sequences for decoupling all interactions ($\overline{H} = 0$) in a spin system. The RL algorithm by itself was only effective at identifying high-fidelity pulse sequences for short sequences. After introducing additional constraints to the algorithm's tree search, high-fidelity pulse sequences were found for all sequence lengths considered. And by introducing common errors in the spin system simulations during training, the algorithm found pulse sequences that were robust to those errors. However, the RL-based pulse sequences did not outperform the CORY48 pulse sequence in computational simulations or experiment.

% TODO talk about this somewhere?
% sparsity of rewards (only at end of episode, so doesn't get immediate feedback)
% discontinuous reward landscape, sparsity of high rewards (most pulse sequences suck, even when constrained with AHT0)

Although the current RL implementation failed to outperform existing analytical approaches, there are several factors that may be considered for improvement.
First, the hyperparameters used for the AlphaZero algorithm were largely unchanged from the original publication, with the exception of the exploration noise added during tree search. Hyperparameter tuning specifically for Hamiltonian engineering may improve the algorithm's performance and increase the fidelities of the resulting pulse sequences. In particular, changing the exploration rate and the training rate relative to data collection from tree search may be fruitful paths to explore.

In addition to optimizing the RL algorithm, using a different set of constraints in the tree search may be beneficial. The $6\tau$ total decoupling constraint that was implemented is not followed in other pulse sequences, including CORY48, so it is possible that decoupling all interactions every $6\tau$ is too strong a constraint on pulse sequences. Two obvious constraints to consider are decoupling dipolar interactions only every $9\tau$, and requiring the pulse sequence use equal numbers of $X, \overline{X}, Y, \overline{Y}$ pulses (both of which are done in the CORY48 sequence). Satisfying higher-order terms in the average Hamiltonian with additional constraints would also be beneficial.

There are other interesting avenues to consider to improve the RL algorithm. One of those avenues is curriculum learning, where the agent learns to solve progressively more difficult tasks (\cite{narvekar2020curriculum}). For Hamiltonian engineering, this may be done by starting with short pulse sequences, and progressively constructing longer pulse sequences.

The spin system simulations for training in the RL algorithm and evaluation of pulse sequences were in the short-$\tau$ delay regime ($d\tau \ll 1$), which improves the ability of the pulse sequences to decouple interactions. However, this is not always realized in experiment, where $d\tau \approx 0.1$ or greater. It would be worth exploring training with longer $\tau$ delays and see how both computational and experimental results change.

Although the primary focus for this work is Hamiltonian engineering, there are many other quantum control problems that could be approached using RL algorithms. Such problems include state-to-state transfer, quantum gate implementation, or bath engineering.
% TODO cite examples of ^, Bukov, etc...


\end{document}
